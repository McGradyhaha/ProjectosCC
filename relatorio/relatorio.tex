\documentclass{llncs}

% hifenização e outras especificações para português
\usepackage[portuguese]{babel}

% hiperligações
\usepackage{hyperref}
\hypersetup{colorlinks=true, urlcolor=blue, linkcolor=black}

% escrever acentos e coisas do género sem que o latex se desoriente
\usepackage[utf8]{inputenc}

% para ter imagens, depois define a directoria de imagens
\usepackage{graphicx}
\graphicspath{{./imagens/}}

\usepackage[labelformat=simple]{caption}
\usepackage[labelformat=empty]{subcaption}

% para ter a informação de quantas páginas tem o documento
\usepackage{lastpage}

% definir o cabeçalho e rodapé
\usepackage{fancyhdr}
\pagestyle{fancy}
\fancyhead[L]{\small{Trabalho Prático Nº2}}
\fancyhead[R]{\small{Comunicações por Computador}}

% ter enumerações alinhadas
\usepackage{enumitem}

% escrever algoritmos
\usepackage[algoruled]{algorithm2e}

% mais cores predefinidas
\usepackage[usenames,dvipsnames]{color}

% definir comandos especiais
\newcommand\doubleplus{+\kern-1.3ex+\kern0.8ex} %

\newcommand{\todo}[1] {\textcolor{BrickRed}{\begin{quote}#1\end{quote}}}



%%%%%%%%%%%%%%%%%%%%%%%%%%%%%%%%%%%%%%%%%%%%%%
%% inicio do documento
\begin{document}

\title{Licenciatura Engenharia Informática	\\  Comunicação por Computadores  \\ Trabalho Prático Nº2}

\author{Bruno Ferreira,Cláudia Oliveira, Duarte Duarte, Fábio Gomes}

\institute{Universidade do Minho}


\maketitle


\begin{abstract}

O presente trabalho foi desenvolvido no âmbito da Unidade Curricular de Comunicações por Computador e tem como principal objetivo a implementação de um serviço de mensagens curtas sobre IPv6. Onde a implementação assenta em dois protocolos: um para descobrir rotas e os nosso vizinhos e o envio de mensagens a esse mesmos vizinhos.

\end{abstract}


\begin{tabular}{ll}
\textbf{Palavras-Chaves} AdHoc, JAVA, Core, UDP
\end{tabular}


\smallskip



%\tableofcontents
%\listoffigures 

\section{Introdução}
	Quando temos uma rede estruturada temos que todos os nós que a ela pertencem possuem um papel especial pois são eles que entre si criam o chamado "mapa" da rede de modo a que cada nó fique a saber como é que a rede se encontra distribuída.
	
	Neste trabalho é pretendido que seja criada uma rede \textit{AdHoc}, onde todos os nós comunicam entre si de modo a termos por exemplo um no A e onde este conhece todos os seus vizinhos.
	
	É pretendido que com desenvolvimento desta aplicação, seja possível o envio de mensagens curtas entre dois nós específicos.

	A nossa aplicação será desenvolvida em JAVA e o teste da topologia utilizada será testada usando o core.

\newpage
\section{Especificação do Protocolo}
O conceito de Rede \textit{AdHoc} remonta as décadas de 70 quando o exercito americano viu a necessidade de comunicar via rádio num ambiente militar, onde a função base era a mobilidade e a comunicação dos dispositivos.
	Atualmente as mesmas redes não são apenas usadas como fins militares mas também como conferências e busca e salvamento.
	O fato de tais redes terem tanto interesse nelas deve-se a estas não necessitarem de uma estrutura fixa para poderem funcionar o que leva a uma rápida implementação.
	Esta situação torna-se vantajosa quando a implementação da rede não pode estar "assente" em infraestruturas devido a segurança, custo, etc..
	Assim, uma rede de computadores ad hoc é aquela na qual todos os terminais funcionam como routers enviando a mensagem que recebe para nós vizinhos que conheça.
	O protocolo que iremos desenvolver irá assentar neste pressupostos, ou seja, a aplicação terá N nodos espalhados por uma área em que cada nó conhece nós que se encontram perto de si.
	Quando um nó recebe uma mensagem que não é para si mas para um dos seus vizinhos encaminha-a para esse vizinho.
	Utilizamos o Core para simular uma rede ad hoc facilitando assim a fase de teste não precisando assim de uma rede física para testar se a aplicação funciona e identificar possíveis lacunas.


\subsection{Primitivas de Comunicação}
As primitivas que foram implementadas foram: \texttt{Hello}, o \texttt{Route Request} e o \texttt{Message} onde este 3 no final permitem criar o funcionamento de uma rede Ad Hoc.


\subsection{Formato das Mensagens Protocolares(PDU)}
O formato que as nossas mensagens protocolares usam para serem enviadas de um nó para outro é o seguinte:

Estas são enviadas com os seguintes campos:
\begin{itemize}
\item Origem - String que identifica a origem de onde sai a mensagem,
\item Destino - String que identifica qual o router ao qual a mensagem se destina;
\item Texto - String que corresponde ao corpo da mensagem;
\item Nodos - Nós por onde a mensagem vai passar (Previamente calculados como RouteRequestPacket).
\end{itemize}


\subsection{Interações}
As iterações que nós temos são um nó possui uma tabela com os seus vizinhos, onde daqui pode fazer as seguintes operações:
\begin{itemize}
\item Enviar mensagens - Premite enviar um mensagem para um nós especifico da rede;
\item Enviar um route reply - Resposta que um nó recebe quando envia um route reuqest ;
\item Enviar um route request - Serve para "perguntar" à rede qual a rota necessária para chegar do meu nó até ao nodo X.
\end{itemize}

Isto são as operações que um determinado router pode fazer com outros routers, mas no entanto este encontra-se sempre a "fazer operações" nas suas tabelas de vizinhos.

As operações que este faz nessa tabela são:
\begin{itemize}
\item Escuta - é a ação que é executada quando a rede se encontra em funcionamento isto é o router possui a rede mas está à espera de que esta mude a qualquer momento dai ele ter esta opção para testar passado um x tempo o estado atual da rede;
\item Manutenção - A manutenção da tabela está relacionado com o fato de ser necessário em alguns dos caso atualizar a tabela de vizinho, isto é, podemos ter caso em que temos de remover nós que se lá encontravam ou netão adicionar novo nós que apareceram na rede ;
\item Consulta - Serve para que quando uma mensagem chega ao router mas no entanto esta não é para ele, ele pode usa a tabela para determinar o próximo router.
\end{itemize}

\section{Implementação}

Durante a fase de implementação surgem  sempre aquelas dúvidas de se a nossa implementação atual é a melhor se o tipo que estamos a usar serve na melhor maneira para representar o que pretendemos.

Por isso começámos por definir o protocolo Hello, pois é onde fazemos a "inicialização" da rede isto é, criamos as tabelas que cada nó possui e à medida que este vai começando a preenche-las começa a saber quais os vizinhos de cada nó.

Depois temos o Route onde se desmembra e request de pedir informação e de reply responde a pergunta que foi feita.

O message é a definição de como é que construimos os datagramas UDP que são enviados entre routers.

\subsection{Detalhes, Parâmetros, Bibliotecas de Funções}
As bibliotecas que foram utilizadas para o desenvolvimento do projeto foram as que permitem trabalhar com datagramas UDP.
Usamos as funções de ArrayLists e HashMaps para podermos processar o armazenamento dos vizinhos de um determinado ip.
Entre outras bibliotecas que permitem tratar o envio de UDP numa rede.

API que utilizamos para a implementação do projeto:
\begin{verbatim}
--HelloPacket--
public HelloPacket(ArrayList<String> vizinhos);
public ArrayList<String> getVizinhos();
public String toString();
public HelloPacket getHelloPacket();
public RouteReplyPacket getRouteReplyPacket();
public RouteRequestPacket getRouteRequestPacket();

--HelloListener--
public HelloListener(HelloTable tabela, MulticastSocket mSocket);
public HelloListener(HelloTable tabela, MulticastSocket mSocket, int id);
private void handleHelloPacket(DatagramPacket recv, HelloPacket pacote);
public void handleRouteRequest(DatagramPacket recv, RouteRequestPacket routereq);
public void run();

--HelloMain--
 public static void main(String[] args) throws Exception;

--HelloMaintenance--
public class HelloMaintenance extends Thread();
public void run();

--HelloMulticaster--
public HelloMulticaster(HelloTable tabela);
public void run();

--HelloPacket--
public HelloPacket(ArrayList<String> vizinhos);
public ArrayList<String> getVizinhos();
public String toString();
public HelloPacket getHelloPacket();
public RouteReplyPacket getRouteReplyPacket();
public RouteRequestPacket getRouteRequestPacket();

--HelloTable--
public HelloTable();
public synchronized void novaEntrada
	(String origem, ArrayList<String> novos) throws IOException;
public synchronized void print();
public synchronized void removerPerdidos();

--Message--
public Message(String origem, String destino, String texto);

--Messages--
public static void addMessage(Message m);
public static void routeMessage
	(String origem, String destino, ArrayList<String> nodos);

--RequestMain--
public RequestMain(HelloTable tabela);
public void run();

--RouteReplyPacket
public RouteReplyPacket(ArrayList<String> nodos);
public ArrayList<String> getRota();
public String getAtual() ;
public boolean sendReply();
public int getNodoAtual();
public String toString();
public HelloPacket getHelloPacket();
public RouteReplyPacket getRouteReplyPacket();
public RouteRequestPacket getRouteRequestPacket();


--RouteRequestPacket--
public RouteRequestPacket
	(ArrayList<String> rota, int nsaltos, int maxsaltos, String destino);
public RouteRequestPacket();
public RouteRequestPacket(String destino);
public RouteRequestPacket(RouteRequestPacket pacote);
public boolean isForMe();
public static boolean sendRequest();
public boolean isForMe();
public String getDestino();
public int getMaxSaltos();
public int getNsaltos();
public String toString();
public HelloPacket getHelloPacket();
public RouteReplyPacket getRouteReplyPacket();
public RouteRequestPacket getRouteRequestPacket();
public void addMeToRoute();


--UnknowPaket--
public abstract HelloPacket getHelloPacket();
public abstract RouteReplyPacket getRouteReplyPacket();
public abstract RouteRequestPacket getRouteRequestPacket();

\end{verbatim}

Os parâmetros que as funções recebem estão diretamente relacionados com a sua execução, por exemplo ao termos o envio de uma mensagem os parametros que esta recebe são a origem da mensagem, o destino da mesma e os nodos para onde estamos a enviar a mensagem.
\section{Testes e Resultados}

\section{Conclusões e Trabalho Futuro}

Após a conclusão deste porjeto, podemos concluir que grande parte dos seus objetivos iniciais foram cumpridos com sucesso.

A implementação numa primeria fase passou pela implementação do protocolo Hello, onde o tipo das mensagens enviadas era simples strings para teste.
Depois seguiram-se mais dois protocolos os de route request que ia "analisar" a rede para fazer a atualização de novos nós ou a de perda de nós e por fim temos o protocolo de mensagens que é onde codificamos a estruturas da mensagens que seriam enviadas e processadas para chegarem do nó origem até ao nó destino.

Como trabalho futuro poderia ser a implementação de uma interface gráfica que facilita a interpretação e interação com a mesma seria também uma mais valia.




\begin{thebibliography}{1}
\bibitem{AD}
http://en.wikipedia.org/wiki/Ad\_hoc
\bibitem{PDU}

http://pt.wikipedia.org/wiki/User\_Datagram\_Protocol

\bibitem{DATA}
http://download.java.net/jdk7/archive/b123/docs/api/java/net/DatagramSocket.html?is-external=true


\end{thebibliography}
\end{document}