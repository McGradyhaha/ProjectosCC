\documentclass[11pt, a4paper, oneside]{article}

% hifenização e outras especificações para português
\usepackage[portuguese]{babel}

% hiperligações
\usepackage{hyperref}
\hypersetup{colorlinks=true, urlcolor=blue, linkcolor=black}

% escrever acentos e coisas do género sem que o latex se desoriente
\usepackage[utf8]{inputenc}

% para ter imagens, depois define a directoria de imagens
\usepackage{graphicx}
\graphicspath{{./imagens/}}

\usepackage[labelformat=simple]{caption}
\usepackage[labelformat=empty]{subcaption}

% para ter a informação de quantas páginas tem o documento
\usepackage{lastpage}

% definir o cabeçalho e rodapé
\usepackage{fancyhdr}
\pagestyle{fancy}
\fancyhead[L]{\small{Trabalho Prático Nº2}}
\fancyhead[R]{\small{Comunicações por Computador}}

% ter enumerações alinhadas
\usepackage{enumitem}

% escrever algoritmos
\usepackage[algoruled]{algorithm2e}

% mais cores predefinidas
\usepackage[usenames,dvipsnames]{color}

% definir comandos especiais
\newcommand\doubleplus{+\kern-1.3ex+\kern0.8ex} %

\newcommand{\todo}[1] {\textcolor{BrickRed}{\begin{quote}#1\end{quote}}}



%%%%%%%%%%%%%%%%%%%%%%%%%%%%%%%%%%%%%%%%%%%%%%
%% inicio do documento
\begin{document}


\author{
  Bruno Ferreira\\
  {\small A61055}
  \and
  Cláduia Oliveira\\
  {\small A60987}\\
  \and
   Duarte Duarte\\
  {\small A61001}\\
  \and
  Fábio Gomes\\
  {\small A61065}\\
}

\title{Licenciatura Engenharia Informática	\\  Comunicação por Computadores  \\ Trabalho Prático Nº2}
\date{\today \\ Universidade do Minho}

\maketitle


\smallskip
\noindent \textbf{Resumo} O presente trabalho foi desenvolvido no âmbito da Unidade Curricular de Comunicações por Computador e tem como principal objetivo a implementação de um serviço de mensagens curtas sobre IPv6. Onde a implementação assenta em dois protocolos: um para descobrir rotas e os nosso vizinhos e o envio de mensagens a esse mesmos vizinhos.

\smallskip
\noindent \textbf{Palavras-Chaves} AdHoc, JAVA, Core, UDP

%\tableofcontents
%\listoffigures 

\section{Introdução}
	Quando temos uma rede estruturada temos que todos os nós que a ela pertencem possuem um papel especial pois são eles que entre si criam o chamado "mapa" da rede de modo a que cada nó fique a saber como é que a rede se encontra distribuída.
	
	Neste trabalho é pretendido que seja criada uma rede \textit{AdHoc}, onde todos os nós comunicam entre si de modo a termos por exemplo um no A e onde este conhece todos os seus vizinhos.
	
	É pretendido que com desenvolvimento desta aplicação, seja possível o envio de mensagens curtas entre dois nós específicos.

	A nossa aplicação será desenvolvida em JAVA e o teste da topologia utilizada será testada usando o core.

\newpage
\section{Especificação do Protocolo}
O conceito de Rede \textit{AdHoc} remonta as décadas de 70 quando o exercito americano viu a necessidade de comunicar via rádio num ambiente militar, onde a função base era a mobilidade e a comunicação dos dispositivos.
	Atualmente as mesmas redes não são apenas usadas como fins militares mas também como conferências e busca e salvamento.
	O fato de tais redes terem tanto interesse nelas deve-se a estas não necessitarem de uma estrutura fixa para poderem funcionar o que leva a uma rápida implementação.
	Esta situação torna-se vantajosa quando a implementação da rede não pode estar "assente" em infraestruturas devido a segurança, custo, etc..
	Assim, uma rede de computadores ad hoc é aquela na qual todos os terminais funcionam como routers enviando a mensagem que recebe para nós vizinhos que conheça.
	O protocolo que iremos desenvolver irá assentar neste pressupostos, ou seja, a aplicação terá N nodos espalhados por uma área em que cada nó conhece nós que se encontram perto de si.
	Quando um nó recebe uma mensagem que não é para si mas para um dos seus vizinhos encaminha-a para esse vizinho.
	Utilizamos o Core para simular uma rede ad hoc facilitando assim a fase de teste não precisando assim de uma rede física para testar se a aplicação funciona e identificar possíveis lacunas.


\subsection{Primitivas de Comunicação}

\subsection{Formato das Mensagens Protocolares(PDU)}
\subsection{Interações}

\section{Implementação}
\subsection{Detalhes, Parâmetros, Bibliotecas de Funções}

\section{Testes e Resultados}
\section{Conclusões e Trabalho Futuro}

\end{document}